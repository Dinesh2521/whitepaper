\begin{abstract}
This paper describes BigchainDB. BigchainDB fills a gap in the decentralization ecosystem: a decentralized
database, at scale. It points to performance of 
1~million writes per second throughput, storing petabytes of data, and sub-second latency.
The BigchainDB design starts with a distributed database (DB), and through a set of
innovations adds blockchain characteristics: decentralized control, immutability, and creation \& movement
of digital assets. BigchainDB inherits characteristics of modern distributed databases: linear scaling in
throughput and capacity with the number of nodes, a full-featured NoSQL query language, efficient
querying, and permissioning. Being built on an existing distributed DB, it also inherits enterprise-hardened
code for most of its codebase. Scalable capacity means that legally binding contracts and certificates may
be stored directly on the blockchain database. The permissioning system enables configurations ranging
from private enterprise blockchain databases to open, public blockchain databases. BigchainDB is
complementary to decentralized processing platforms like Ethereum, and decentralized file systems like
InterPlanetary File System (IPFS). This paper describes technology perspectives that led to the BigchainDB
design: traditional blockchains, distributed databases, and a case study of the domain name system (DNS).
We introduce a concept called blockchain pipelining, which is key to scalability when adding blockchain-like
characteristics to the distributed DB. We present a thorough description of BigchainDB, an
analysis of latency, and preliminary experimental results. The paper concludes with a description of use cases.

\medskip

\textbf{This is no longer a living document. All changes made since June~8, 2016 are noted in an Addendum attached at the end.}

\end{abstract}
