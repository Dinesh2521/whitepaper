\section{BigchainDB Deployment}\label{sec:deployment}

\subsection{BigchainDB Use Cases}

Many BigchainDB use cases are like traditional blockchain use cases, except focused on situations where higher throughput, lower latency, or more storage capacity is necessary; where powerful querying or permissioning are helpful; or for simplicity of deployment since it feels like using a NoSQL database.
For example, BigchainDB can handle the throughput of high-volume payment processors, and directly store contracts receipts, or other related documents on the DB alongside the actual transaction.
Another example is “Bitcoin 2.0” applications, to keep transaction costs reasonable as the application scales\footnote{
As of February 2016, the fee required to ensure that a Bitcoin transaction goes through is about \$$0.10$~USD. There is no predefined fee for Bitcoin transactions; it's determined by market forces.
To estimate the fee, we can look at the fees that were paid in the last X blocks, and use that information to calculate a fee that will give a high probability that a transaction will go into the next block.
Usually a higher fee will give a higher priority to a transaction, but in the end, it's the miners that decide whether they want to include the transaction in the block or not.
Usually the fees are around $15$k-$20$k satoshi per $1000$ bytes, so that would be the average fee to pay.
Useful resources are blockchain.info \cite{blockchaininfo2015transaction_fees}, Tradeblock \cite{tradeblock} (in the charts, select fee/size), and the Bitcoin wiki page on transaction fees \cite{bitcoin2015fees}.
}.

Some BigchainDB use cases are also like traditional distributed DB use cases, except focused where blockchain characteristics can benefit: decentralization, immutability, and the ability to create and transfer digital assets. For example, improving DB reliability by not having a single point of failure, or storage of documents with secure time-stamping.

BigchainDB use cases include:
\begin{itemize}
 \item \textbf{Tracking intellectual property assets} along the licensing chain. BigchainDB can reduce licensing friction in channels connecting creators to audiences, and gives perfect provenance to digital artifacts. A typical music service has $38$ million songs—BigchainDB could store this information in a heartbeat, along with licensing information about each song and information about use by subscribers. In another example, consider a medium-sized photo marketplace running $100,000$ transactions a day; to put this on Bitcoin would cost \$$10,000$ per day and tie up the Bitcoin network.
 \item \textbf{Time stamping, receipts, and certification.} BigchainDB reduces legal friction by providing irrefutable evidence of an electronic action. And, BigchainDB is big enough that supporting information like receipts and certificates of authenticity (COAs) can be stored \textit{directly} on it, rather than linking to the document or storing a hash.
 \item \textbf{Legally-binding contracts} can be stored directly on the BigchainDB next to the transaction, in a format that is readable by humans and computers \cite{grigg2004ricardian}.
 \item \textbf{Creation and real-time movement of high-volume financial assets.} Only the owner of the asset can move the asset, rather than the network administrator like in previous database systems. This capability reduces costs, minimizes transaction latency, and enables new applications.
 \item \textbf{Tracking high-volume physical assets} along whole supply chain. BigchainDB can help reduce fraud, providing massive cost savings. Every RFID tag in existence could be entered on a BigchainDB.
 \item \textbf{Smart contracts} (decentralized processing), where the application must be fully decentralized and database functionality is crucial.
 \item \textbf{Data science} applications where the BigchainDB captures massive data streams, and data scientists can easily run queries against BigchainDB in their data mining and analytics tasks.
 \item \textbf{Improving database reliability} by creating resistance to single points of failure. This reliability helps move past the status quo where a single hack leads to massive data loss, like in Target, Sony, or the OPM \cite{ojeda2014target_hack}\cite{bluestone2014sony_hack}\cite{davis2015hacking}.
\end{itemize}

\subsection{Transaction Encryption}
Normally, transactions stored in BigchainDB aren't encrypted, but users can encrypt the payload if they want, using the encryption technology of their choice. (The payload of a transaction can be any valid JSON, up to some maximum size as explained below.) Other aspects of the transaction, such as the current owner's public key and the new owner's public key, aren't encrypted and can't be encrypted.

\subsection{BigchainDB Limitations}
Because BigchainDB is built on top of an existing ``big data'' database, it inherits many of the limitations of that database.

The first version of BigchainDB is built on top of RethinkDB, so it inherits some of RethinkDB's limitations, including a maximum of 32 shards per table (increasing to 64). While there's no hard limit on the document (transaction) size in RethinkDB, there is a recommended limit of 16MB for memory performance reasons. Large files can be stored elsewhere; one would store only the file location or hash (or both) in the transaction payload.

\subsection{BigchainDB Products \& Services}
We envision the following products and services surrounding BigchainDB:
\begin{enumerate}
 \item \textbf{BigchainDB: a blockchain database} with high throughput, high capacity, low latency, rich query support, and permissioning.
 \begin{itemize}
  \item For large enterprises and industry consortia creating new private trust networks, to take advantage of blockchain capabilities at scale or to augment their existing blockchain deployments with querying and other database functionality
  \item BigchainDB will be available in an out-of-the-box version that can be deployed just like any other DB, or customized versions (via services, or customized directly by the user).
  \item BigchainDB will include interfaces such as a REST API, language-specific bindings (e.g. for Python), RPC (like bitcoind), and command line. Below that will be an out-of-the-box core protocol, out-of-the-box asset overlay protocol \cite{dejonghe_spool}, and customizable overlay protocols.
  \item BigchainDB will support legally binding contracts, which are generated automatically and stored directly, in a format readable by both humans and computers \cite{grigg2004ricardian}. There will be out-of-box contracts for out-of-the-box protocols, and customizable contracts for customizable protocols.
  \item BigchainDB will offer cryptographic COAs, which can be generated automatically and stored directly on the BigchainDB. There will be out-of-box and customizable versions.
  \item BigchainDB is built on a large, open-source pre-existing database codebase that has been hardened on enterprise usage over many years. New code will be security-audited and open source.
 \end{itemize}
 \item \textbf{BigchainDB as a Service}, using a public BigchainDB instance, or a private BigchainDB with more flexible permissioning.
 \begin{itemize}
  \item For developers who want the benefits of blockchain databases without the hassle of setting up private networks.
  \item For cloud providers and blockchain platform providers who want scalable blockchain database as part of their service.
  \item For “Bitcoin 2.0” companies looking to keep transaction costs reasonable as they go to scale
  \item Main interfaces will be a REST API directly, REST API through cloud providers, and language-specific bindings (e.g. Python).
  \item With the features of the BigchainDB listed above.
 \end{itemize}
 \item \textbf{“Blockchain-ify your database”} service, to help others bring blockchain properties to other distributed DBs. Think MySqlChain, CassandraChain, and Neo4jChain.
 \begin{itemize}
  \item For DB vendors looking to extend their DB towards blockchain applications.
  \item For DB developers who want to play with blockchain technology.
 \end{itemize}
\end{enumerate}


\subsection{Timeline}
Like many, we have known about Bitcoin and blockchain scalability issues for years.
Here’s the timeline of how BigchainDB took shape:
\begin{itemize}
  \item Oct 2014 – Gave our first public talk on big data and blockchains \cite{mcconaghy2014blockchain}
  \item Apr 2015 – Preliminary investigations; paused the project to focus on our IP business
  \item Sep 2015 – Re-initiated the project; detailed design; building and optimizing
  \item Dec 2015 – Benchmark results of $100,000$ transactions/s
  \item Dec 2015 – Alpha version of the software integrated into an enterprise customer prototype
  \item Dec 2015 – Initial drafts of the whitepaper shared with some reviewers
  \item Jan 2016 – Benchmark results of $1,000,000$ transactions/s
  \item Feb 10, 2016 – BigchainDB was publicly announced
  \item Feb 10, 2016 – The first public draft of the whitepaper was released
  \item Feb 10, 2016 – Version 0.1.0 of the software was released open-source on GitHub\footnote{\url{http://github.com/bigchaindb/bigchaindb}. Core BigchainDB software is licensed under an Affero GNU Public License version 3 (AGPLv3). The BigchainDB drivers supported by ascribe GmbH are licensed under an Apache License (version 2.0).}. The software was not recommended for external use yet, but development was in the open.
  \item Apr 26, 2016 – Version 0.2.0 released
\end{itemize}

The current BigchainDB Roadmap can be found in the \texttt{bigchaindb/org} repository on GitHub\footnote{\url{https://github.com/bigchaindb/org/blob/master/ROADMAP.md}}.