\section{Other Experiments}\label{appendix:benchmarks}

\subsection{Experimental Setup}\label{subsec:benchmarks_setup}

To test the writing performance, we created a process that inserts a block in the database in an infinite loop.

The block is a valid block with small transactions.
In our case, we used valid transactions without any payload. 
An entire block occupies about $900$KB.

\begin{minipage}{\linewidth}
  \begin{lstlisting}[style=python]
  while True:
    r.table(table).insert(r.json(BLOCK_SERIALIZED), durability='soft').run(conn)\end{lstlisting}
\end{minipage}

In $\mathtt{hard}$ durability mode, writes are committed to disk before acknowledgments are sent; in $\mathtt{soft}$ mode, writes are acknowledged immediately after being stored in memory.

This means that the insert will block until RethinkDB acknowledges that the data was cached.
In each server we can start multiple processes.

\medskip
\noindent\textbf{Write Units.} Let’s define \textit{$1$ write unit} as being 1 process.
For example, in a $32$ node cluster, with each node running $2$ processes, we would have $64$ write units.
This will make it easier to compare different tests.

\medskip
\noindent\textbf{Sharding} in distributed datastores means partitioning a table so that the data can be evenly distributed between all nodes in the cluster.
In most distributed datastores, there is a maximum number of shards per table. For RethinkDB, that limit is $32$ shards per table.

In RethinkDB, a shard is also called a primary replica, since by default the replication factor is $1$.
Increasing the replication factor produces secondary replicas that are used for data redundancy. If a node holding a primary replica goes down, another node holding a secondary replica of the same data can step up and become the primary replica.

\medskip
\noindent\textbf{Compute Resources.} For these tests we are using $32$-core AWS EC2 instances with SSD storage and 10Gbps network connections ($\mathtt{c3.8xlarge}$).
For the tests, we used either $32$- or $64$-node clusters all running in the same AWS region.

\subsection{Experiments on Throughput}
The experimental setup is like the setup described in section \ref{subsec:benchmarks_setup}. 

\subsubsection{Experiment 1}\label{subsubsec:appendix_benchmarks_exp1}

\medskip
\noindent\textbf{Settings:}
\begin{itemize}
 \item \textbf{Number of nodes:} $32$
 \item \textbf{Number of processes:} $2$ processes per node
 \item \textbf{Write units:} $32 \times 2 = 64$ write units
\end{itemize}

\medskip
\noindent\textbf{Results:}
\begin{itemize}
 \item \textbf{Output:} stable $1$K writes per second
\end{itemize}

This was the most successful experiment. We were able to reach a stable output of $1$K blocks per second. The load on the machines is stable and the IO is at an average of $50-60 \%$.

Other tests have shown that increasing the number write units per machine can lead to a stable performance up to $1.5$K writes per second but the load on the nodes would increase until the node would eventually fail. 
This means that we are able to handle bursts for a short amount of time ($10-20$ min).

This test can be used has a baseline for the future in where $64$ writes equal $1$K transactions per second. 
Or, that each write unit produces an output of $(1000/64) \approx 64$ writes per second.

\subsubsection{Experiment 2}\label{subsubsec:appendix_benchmarks_exp2}
\medskip
\noindent\textbf{Settings:}
\begin{itemize}
 \item \textbf{Number of nodes:} $32$
 \item \textbf{Number of processes:}
 \begin{itemize}
  \item$16$ nodes running $2$ processes
  \item$16$ nodes running $3$ processes
 \end{itemize}
 \item \textbf{Write units:} $16 \times 3 + 16 \times 2 = 80$ write units
 \item \textbf{Expected output:} $1.25$K writes per second
\end{itemize}

\medskip
\noindent\textbf{Results:}
\begin{itemize}
 \item \textbf{Output:} stable $1.2$K writes per second
\end{itemize}

Increasing a bit the number of write units shows an increase in output close to the expected value but in this case the IO around $90\%$ close to the limit that the machine can handle.

\subsubsection{Experiment 3}\label{subsubsec:appendix_benchmarks_exp3}
\medskip
\noindent\textbf{Settings:}
\begin{itemize}
 \item \textbf{Number of nodes:} $32$
 \item \textbf{Number of processes:}
 \begin{itemize}
  \item$16$ nodes running $2$ processes
  \item$16$ nodes running $4$ processes
 \end{itemize}
 \item \textbf{Write units:} $16 \times 4 + 16 \times 2 = 96$ write units
 \item \textbf{Expected output:} $1.5$K writes per second
\end{itemize}

\medskip
\noindent\textbf{Results:}
\begin{itemize}
 \item \textbf{Output:} stable $1.4$K writes per second
\end{itemize}

This test produces results similar to previous one. 
The reason why we don't quite reach the expected output may be because RethinkDB needs time to cache results and at some point increasing the number of write units will not result in an higher output. 
Another problem is that as the RethinkDB cache fills (because the RethinkDB is not able to flush the data to disk fast enough due to IO limitations) the performance will decrease because the processes will take more time inserting blocks. 

\subsubsection{Experiment 4}\label{subsubsec:appendix_benchmarks_exp4}
\medskip
\noindent\textbf{Settings:}
\begin{itemize}
 \item \textbf{Number of nodes:} $64$
 \item \textbf{Number of processes:} $1$ process per node
 \item \textbf{Write units:} $64 \times 1 = 64$ write units
 \item \textbf{Expected output:} $1$K writes per second
\end{itemize}

\medskip
\noindent\textbf{Results:}
\begin{itemize}
 \item \textbf{Output:} stable $1$K writes per second
\end{itemize}

In this case, we are increasing the number of nodes in the cluster by $2\times$. 
This won't have an impact in the write performance because the maximum amount of shards per table in RethinkDB is $32$ (RethinkDB will probably increase this limit in the future).
What this provides is more CPU power (and storage for replicas, more about replication in the next section). 
We just halved the amount write units per node maintaining the same output. 
The IO in the nodes holding the primary replica is the same has Experiment \ref{subsubsec:appendix_benchmarks_exp1}. 

\subsubsection{Experiment 5}\label{subsubsec:appendix_benchmarks_exp5}
\medskip
\noindent\textbf{Settings:}
\begin{itemize}
 \item \textbf{Number of nodes:} $64$
 \item \textbf{Number of processes:} $2$ processes per node
 \item \textbf{Write units:} $64 \times 2 = 128$ write units
 \item \textbf{Expected output:} $2000$ writes per second
\end{itemize}

\medskip
\noindent\textbf{Results:}
\begin{itemize}
 \item \textbf{Output:} unstable $2$K (peak) writes per second
\end{itemize}

In this case, we are doubling the amount of write units. 
We are able to reach the expected output, but the output performance is unstable due to the fact that we reached the IO limit on the machines. 

\subsubsection{Experiment 6}\label{subsubsec:appendix_benchmarks_exp6}
\medskip
\noindent\textbf{Settings:}
\begin{itemize}
 \item \textbf{Number of nodes:} $64$
 \item \textbf{Number of processes:}
 \begin{itemize}
  \item$32$ nodes running $1$ process
  \item$32$ nodes running $2$ processes
 \end{itemize}
 \item \textbf{Write units:} $32 \times 2 + 32 \times 1 = 96$ write units
 \item \textbf{Expected output:} $1.5$K writes per second
\end{itemize}

\medskip
\noindent\textbf{Results:}
\begin{itemize}
 \item \textbf{Output:} stable $1.5$K writes per second
\end{itemize}

This test is similar to Experiment \ref{subsubsec:appendix_benchmarks_exp3}. 
The only difference is that now the write units are distributed between $64$ nodes meaning that each node is writing to its local cache and we don't overload the cache of the nodes like we did with Experiment \ref{subsubsec:appendix_benchmarks_exp3}. 
This is another advantage of adding more nodes beyond $32$.

\subsection{Experiments on Replication}

Replication is used for data redundancy.
In RethinkDB we are able to specify the number of shards and replicas per table.
Data in secondary replicas is no directly used, it’s just a mirror of a primary replica and used in case the node holding the primary replica fails.

RethinkDB does a good job trying to distribute data evenly between nodes.
We ran some tests to check this.

By increasing the number of replicas we also increase the number of writes in the cluster.
For a replication factor of $2$ we double the amount of writes on the cluster, with a replication factor of $3$ we triple the amount of writes and so on.

With $64$ nodes and since we can only have $32$ shards we have $32$ nodes holding shards (primary replicas)

With a replication factor of $2$, we will have $64$ replicas ($32$ primary replicas and $32$ secondary replicas).
Since we already have $32$ nodes holding the $32$ shards/primary replicas RethinkDB uses the other $32$ nodes to hold the secondary replicas. 
So in a $64$ node cluster with $32$ shards and a replication factor of $2$, $32$ nodes will be holding the primary replicas and the other $32$ nodes will be holding the secondary replicas.

If we run Experiment \ref{subsubsec:appendix_benchmarks_exp4} again with this setup, except now with a replication factor of $2$, we get twice the amount of writes. 
A nice result is that the IO in the nodes holding the primary replicas does not increase when compared to Experiment \ref{subsubsec:appendix_benchmarks_exp4} because all of the excess writing is now being done the $32$ nodes holding the secondary replicas.

Also regarding replication: if I have a $64$ node cluster and create a table with $32$ shards, $32$ nodes will be holding primary replicas and the other nodes do not hold any data. 
If I create another table with $32$ shards RethinkDB will create the shards in the nodes that where not holding any data, evenly distributing the data.
