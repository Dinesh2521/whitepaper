\section{Private vs. Public BigchainDB, and Authentication}\label{sec:permissioning}

\subsection{Introduction}
The way that BigchainDB is designed, permissioning sits at a layer above the core of the design. 
However, we have already seen many questions about “private vs. public” versions of BigchainDB, privacy, and authentication. 
In our view, a rich permissioning framework is the technology foundation. 
This section explores permissions, roles, private BigchainDBs, and privacy. 
It then has an extended section on a public BigchainDB, which we believe is tremendously important. 
It finally discusses authentication and the role of certificate-granting authorities.

\subsection{Permissions, Identities, and Roles}
Permissions are rules about what a user can do with a piece of data. 
Permissions are used in all kinds of computing environments, from shared file systems like Dropbox and Google Drive, to local file systems in Windows, iOS, and Linux, to distributed DBs. 
We should expect blockchain DBs to have rich permissioning systems.

Permissioning ideas from these other systems can inform our design. 
In Unix, each file or directory has three identity roles (owning user, owning group, others) and three types of permissions for each role (read, write, execute), for a total of nine permission values. 
For example, the permission values ``\texttt{rwxr--r---}'' means that the owning user can read, write, and execute (\texttt{rwx}); the owning group can read but not write or execute (\texttt{r--}), and others have no permissions (\texttt{---}).

A BigchainDB database instance is characterized by which identities have which permissions. 
Table \ref{tab:permissioning_enterprise} and Table \ref{tab:permissioning_public} gives examples of permissions on a private and public BigchainDB, respectively. 
This is loosely comparable to a corporation’s internal intranet and the public Internet. 
We will elaborate on these shortly.

\begin{savenotes}
\begin{table}[ht!]
  \caption{Example Permissioning / Roles in an Enterprise BigchainDB Instance}
  \footnotesize
  \makebox[\textwidth][c]{
  \centering
  \setlength\extrarowheight{3pt}
  \begin{tabular}{ | m{\dimexpr 0.2\linewidth-2\tabcolsep} |
	  M{\dimexpr 0.12\linewidth-2\tabcolsep} |
	  M{\dimexpr 0.1\linewidth-2\tabcolsep} |
	  M{\dimexpr 0.1\linewidth-2\tabcolsep} |
	  M{\dimexpr 0.1\linewidth-2\tabcolsep} |
	  M{\dimexpr 0.1\linewidth-2\tabcolsep} |
	  M{\dimexpr 0.1\linewidth-2\tabcolsep} |
	  M{\dimexpr 0.16\linewidth-2\tabcolsep} |
	  M{\dimexpr 0.1\linewidth-2\tabcolsep} |
	  M{\dimexpr 0.12\linewidth-2\tabcolsep} | } \Xhline{4\arrayrulewidth}\rowcolor{black}\color{white}
  Action    & \color{white} Requires vote & \color{white} Voting Node & \color{white} Sys Admin & \color{white} Issuer & \color{white} Trader & \color{white}Broker & \color{white} Authenticator & \color{white} Auditor & \color{white} Core vs Overlay \\\Xhline{4\arrayrulewidth}
  Vote on Admin \& Asset Actions & & Y & & & & & & & \textbf{Core} \\\Xhline{2\arrayrulewidth}\rowcolor{backcolour}
  \multicolumn{10}{|c|}{Admin actions}\\\Xhline{2\arrayrulewidth}
  Update Role or Permissions & Y & Y & Y & & & & & & \textbf{Core} \\\hline
  Add/Remove Voting Node & Y & Y & Y\footnote{\label{note:permissioning_sysadmin}Action is permitted only during the network initiatization process. Once a network is live, the sys admin can no longer act unilaterally.} & & & & & & \textbf{Core} \\\hline
  Update software & Y & Y & Y & & & & & & \textbf{Core} \\\Xhline{2\arrayrulewidth}\rowcolor{backcolour}
  \multicolumn{10}{|c|}{Asset actions}\\\Xhline{2\arrayrulewidth}
  Issue Asset                        & Y & & & Y &   &   &   & & \textbf{Core} \\\hline
  Transfer Asset                     & Y & & & O & O & P &   & & \textbf{Core} \\\hline
  Receive Asset                      & Y & & & Y & Y &   &   & & \textbf{Core} \\\hline
  Grant Read Access on Asset         & Y & & & O & O & P & P & & \textbf{Core} \\\hline
  Consign Asset                      & Y & & & O & O &   &   & & \textbf{Overlay} \\\hline
  Receive Asset Consignment          & Y & & & Y & Y & Y &   & & \textbf{Overlay} \\\hline
  Add Asset Information              & Y & & & O & O & P &   & & \textbf{Overlay} \\\hline
  Add Authentication Information     & Y & & & O & O &   & P & & \textbf{Overlay} \\\hline
  Create Certificate of Authenticity & N & & & O & O & P &   & & \textbf{Overlay} \\\Xhline{2\arrayrulewidth}\rowcolor{backcolour}
  \multicolumn{10}{|c|}{Read actions}\\\Xhline{2\arrayrulewidth}
  Read Asset Information             & N & Y & Y & O & Y & P & P & P & \textbf{Overlay} \\\hline
  Read Certificate of Authenticity   & N & Y & Y & O & Y & P & P & P & \textbf{Overlay} \\
  \Xhline{4\arrayrulewidth}
  \end{tabular}
  }
  \label{tab:permissioning_enterprise}
\end{table}
\end{savenotes}

An \textbf{identity}, which signifies the holder of a unique private key, can be granted a \textbf{permission} for each transaction type. 
Permissions, as reflected on the tables, can be as follows: “\textbf{Y}” means the identity can perform a transaction; “\textbf{O}” means the identity can perform a transaction if the identity is the owner of the asset, which is indicated by holding the private key to that asset; and “\textbf{P}” means can perform a transaction, after the owner of the asset has given permission to the identity. 
Most transactions need to be voted as approved or not approved by voting nodes, with the exception of read operations.

A role is a group of individual permissions. 
Roles facilitate permission assignment and help clarify the rights given to users in their identity and with their permissions. 
Roles can be custom-created depending on the context. 
An identity may hold multiple roles, where the identity’s permissions are the sum of the identity’s role permissions and any other permissions that have been granted to it.

The core BigchainDB protocol includes as few actions or transaction types as possible, in order to maximize backwards-compatibility and minimize complexity. 
Overlay protocols can add new features, such as SPOOL \cite{dejonghe_spool} for unique asset ownership, which adds actions like consignment and authentication to property transactions. 
Table \ref{tab:permissioning_enterprise} and Table \ref{tab:permissioning_public} each have core protocol actions, as well as some overlay protocol actions from SPOOL.

\subsection{Private BigchainDBs}
A private BigchainDB could be set up amongst a group of interested parties to facilitate or verify transactions between them in a wide variety of contexts, such as exchanging of securities, improving supply chain transparency, or managing the disbursement of royalties. 
For instance, the music industry could choose to form a trust network including record labels, musicians, collecting societies, record stores, streaming services, and support providers such as lawyers and communications agencies. 
A consortium of parties operating a private BigchainDB comprise an “enterprise trust network” (ETN).

Table \ref{tab:permissioning_enterprise} illustrates permissioning of a sample private BigchainDB instance for use in an ETN.

The first column shows actions allowed; the second column shows whether the action needs voting; columns 3-9 are roles; and the final column indicates whether the action is part of the core BigchainDB protocol.

An identity holding the “Voting Node” role (column 3) can vote on asset actions. No other role is able to do so.

Voting Nodes can update permissions for any other identity (row 2). 
Changing permissions requires voting consensus from other nodes. 
This is how new identities entering the system are assigned permissions or roles, and also how Voting Nodes are added or removed.

A Voting Node may propose an update to the Voting Node software (row 4).
Voting Nodes will only update once they reach consensus.
Voting Nodes also must be able to read an asset (rows 14-15) in order to be able to vote.

Like a Voting Node, an identity with “Sys Admin” role can propose to update permissions or update voting node software. 
This role is helpful because voting nodes may not be up-to-date technically when the Sys Admin is. 
Crucially, the Sys Admin cannot unilaterally update the software; rather, it can only propose software and ensure that the voting nodes hit consensus about updating. 
The Sys Admin must also be able to read an asset (rows 14-15), in order to debug issues that may arise with the software.

The main job of the “Issuer” role is to issue assets (row 5).
But it can also do everything that a Trader role can do.

The “Trader” role conducts trades of assets, has others conduct trades on its behalf, and lines up reading and authentication of assets.
It can transfer ownership to an identity, though only if it is the owner of the asset as indicated by the “O” (row 6); or be on the receiving end of an asset transfer (row 7).
Similarly, if it is the owner then it can consign an asset to have another identity transact with the asset on its behalf (row 9); or be on the receiving end as consignee (row 10). By default, read permissions are off, but a Trader can allow others to read the asset info (row 10 grants permission; row 15 read). The Trader can also add arbitrary data or files to an asset (row 11).

A “Broker / Consignee” role (column 7) gets a subset of the Trader’s permissions - only what is needed to be able to sell the work on the owner’s behalf.

We describe the “Authenticator” role (column 8) further in section \ref{subsec:permissioning_authentication}.

For simplicity of presentation, some details have been omitted compared to the actual implementation.
For example, usually a Consignee has to accept a consignment request.

\subsection{Privacy}

\noindent \textbf{Q}: Bank A doesn’t want Bank B to see their transactions.
But if they’re both voting nodes, can Bank B see Bank A’s transactions?

\noindent \textbf{A}: This is not really related to BigchainDB; it doesn’t care about the content of the transaction.
For the transaction to be valid, it just needs to have a current unspent input and a correct signature.

\medskip
\noindent \textbf{Q}: But if every voting node knows the identity of the nodes in a transaction, and can see the amount transacted in order to validate the transaction, than isn’t that a loss of privacy?

\noindent \textbf{A}: The nodes in a transaction are just public keys.
The way that the mapping between a public key and an identity is done should be taken care of by Bank A and Bank B.
If they want to hide the amount that is being transacted, they can do that; BigchainDB doesn't care.

Let’s say that Bank A creates an input “A” and gives it to “PUBKA”, and inside the data field it says that this is an input for “B”, and “B” is just a serial number.
BigchainDB makes sure that input “A” can only be spent once and that only “PUBKA” can spend it.
There are no amounts involved.

\subsection{A Public BigchainDB}\label{subsec:permissioning_public}
\subsubsection{Introduction}
A BigchainDB can be configured to be more public, with permissioning such that anyone can issue assets, trade assets, read assets, and authenticate. 
We are taking steps towards a first public BigchainDB\footnote{We envision that ultimately there will be many public BigchainDBs. More than one is not a big problem, as there is no native token built into the DB. Universal resource indicators (URIs) will make them easy to distinguish.}.

\subsubsection{Motivations}

Decentralization technology has potential to enable a new phase for the Internet that is open and democratic but also easy to use and trust \cite{swan2015blockchain}\cite{andreesen2014bitcoin}\cite{monegro2014blockchain_stack}.
It is intrinsically democratic, or at the very least disintermediating.
It is also trustworthy: cryptography enables the conduct of secure and reliable transactions with strangers without needing to trust them, and without needing a brand as proxy.

The discourse is around benefits in both the public and private sector.
In the public sector, the most obvious benefit is in the future shape of the Internet and especially the World Wide Web \cite{berners1989information_mgmnt}.
These technologies have fundamentally reshaped society over the past two decades. The 90s Web started out open, free-spirited, and democratic. 
In the past 15 years, power has consolidated across social media platforms and the cloud.
People around the world have come to trust and rely on these services, which offer a reliability and ease of use that did not exist in the early Internet. 
However, these services are massively centralized, resulting in both strict control by the central bodies and vulnerability to hacking by criminals and nation-states.

Decentralization promises a large positive impact on society. 
An antidote to these centralized services and concentration of power is to re-imagine and re-create our Internet via decentralized networks, with the goal of giving people control over their data and assets and redistributing power across the network.

\subsubsection{Public BigchainDB Roles}

\begin{table}[ht!]
  \caption{Example Permissioning / Roles in an Public BigchainDB}
  \footnotesize
  \makebox[\textwidth][c]{
  \centering
  \setlength\extrarowheight{3pt}
  \begin{tabular}{ | m{\dimexpr 0.3\linewidth-2\tabcolsep} |
	  M{\dimexpr 0.12\linewidth-2\tabcolsep} |
	  M{\dimexpr 0.1\linewidth-2\tabcolsep} |
	  M{\dimexpr 0.1\linewidth-2\tabcolsep} |
	  M{\dimexpr 0.1\linewidth-2\tabcolsep} |
	  M{\dimexpr 0.1\linewidth-2\tabcolsep} |
	  M{\dimexpr 0.17\linewidth-2\tabcolsep} |} \Xhline{4\arrayrulewidth}\rowcolor{black}\color{white}
  Action    & \color{white} Requires vote & \color{white} Voting Node & \color{white} Sys Admin & \color{white} Issuer & \color{white} User & \color{white} Authenticator \\\Xhline{4\arrayrulewidth}
  Vote on Admin \& Asset Actions & & Y & & & & \textbf{Core} \\\Xhline{2\arrayrulewidth}\rowcolor{backcolour}
  \multicolumn{7}{|c|}{Admin actions}\\\Xhline{2\arrayrulewidth}
  Update Role or Permissions & Y & Y & Y & & & \textbf{Core} \\\hline
  Add/Remove Voting Node & Y & Y & Y\footnotemark[\getrefnumber{note:permissioning_sysadmin}] & & & \textbf{Core} \\\hline
  Update software & Y & Y & Y & & & \textbf{Core} \\\Xhline{2\arrayrulewidth}\rowcolor{backcolour}
  \multicolumn{7}{|c|}{Asset actions}\\\Xhline{2\arrayrulewidth}
  Issue Asset                        & Y & & & Y &   & \textbf{Core} \\\hline
  Transfer Asset                     & Y & & & O &   & \textbf{Core} \\\hline
  Receive Asset                      & Y & & & Y &   & \textbf{Core} \\\hline
  Grant Read Access on Asset         & Y & & & N/A & N/A & \textbf{Core} \\\hline
  Consign Asset                      & Y & & & O &   & \textbf{Overlay} \\\hline
  Receive Asset Consignment          & Y & & & Y &   & \textbf{Overlay} \\\hline
  Add Asset Information              & Y & & & O &   & \textbf{Overlay} \\\hline
  Add Authentication Information     & Y & & & O & P & \textbf{Overlay} \\\hline
  Create Certificate of Authenticity & N & & & O &   & \textbf{Overlay} \\\Xhline{2\arrayrulewidth}\rowcolor{backcolour}
  \multicolumn{7}{|c|}{Read actions}\\\Xhline{2\arrayrulewidth}
  Read Asset Information             & N & Y & Y & O & P & \textbf{Overlay} \\\hline
  Read Certificate of Authenticity   & N & Y & Y & O & P & \textbf{Overlay} \\
  \Xhline{4\arrayrulewidth}
  \end{tabular}
  }
  \label{tab:permissioning_public}
\end{table}

Table \ref{tab:permissioning_public} describes permissioning of a public BigchainDB instance.
Here, BigchainDB is configured such that each User (column 6) can do anything, except for sensitive roles such as voting, administration, and authentication.
Critically, Users can issue any asset (column 6, row 5) and read all assets (column 6, row 14); this is one of the defining features of an open blockchain. 

\subsubsection{Public BigchainDB Federation Caretakers}

At the core of a public BigchainDB are the “Caretakers”: organizations with an identity that has a “Voting Node” role.
An identity with that role can vote to approve or reject transactions, and can vote whether to assign the “Voting Node” role to another identity. 
(Note: an organization can have more than one identity, so a Caretaker could have two or more identities with a “Voting Node” role.)

To start, the public BigchainDB will have five identities with the Voting Node role: three held by ascribe and two held by other organizations chosen by ascribe.
That is, the public BigchainDB will start with three Caretakers: ascribe and two others.
From there, additional Caretakers will be selected and added to the federation by existing Caretakers.
Caretakers will have divergent interests to avoid collusion, but must have one thing in common: they must have the interests of the Internet at heart.
In choosing Caretakers, there will be a preference to organizations that are non-profit or building foundational technology for a decentralized Internet; and for diversity in terms of region, language, and specific mandate.

The right organizational structure will be critical to the success of a public BigchainDB.
Governance issues have plagued the Bitcoin blockchain \cite{popper2016bitcoin_crisis}.
We can take these as lessons in the design of a public BigchainDB.
We are consulting with lawyers, developers, academics, activists, and potential Caretakers to develop a strong, stable system that is transparent enough to be relied on and flexible enough to meet the needs of the network.

Ultimately, the public BigchainDB will operate entirely independently under its own legal entity.
It will choose its own Caretakers and set its own rules—but it will always work toward the long-term goal of a free, open, and decentralized Internet.

\medskip

\fbox{\begin{minipage}{\linewidth}
\centering
\medskip
We are gathering a consortium of potential public BigchainDB nodes. If you think your organization fits as one of these nodes, or have suggestions for nodes, please contact the authors.
\medskip
\end{minipage}}

\subsection{BigchainDB Authentication of Assets}\label{subsec:permissioning_authentication}

The “Authenticator” role gives a formal place for authentication and certificate-granting authorities.
Examples may include a credit rating agency, an art expert certifying the authenticity of a painting, a university issuing a degree, a governmental body issuing a permit, or a notary stamping a document.

While the BigchainDB can function completely without the Authenticator role, in a decentralized network where anyone can issue assets, it is clear that third parties will step in to provide an extra layer of trust for asset buyers.

These third parties would do all the things trusted third parties do today—act as an escrow agent, place a stamp or seal of approval, issue a certificate, or rate the quality or reputation of the asset issuer.

For authentication to be issued, a Trader enables an identity to have read and authentication permission on an asset (Table \ref{tab:permissioning_enterprise}, row 8), then the Authenticator reviews all relevant information about the asset, and issues a report as a transaction (Table \ref{tab:permissioning_enterprise}, row 12).

The owner of the asset may then create a cryptographic Certificate of Authenticity (COA), a digital document that includes all the digitally signed authentication reports from the various authenticators.
The COA is digitally signed as well, so even if printed out, tampering can be detected.
The COA can then be used by the seller as a pointer to authenticity, to show that the asset is not fraudulent.

A public BigchainDB should not be prescriptive—it should be open to new sources of authority.
The issuance of certificates does not have to be limited to traditional authorities.
BigchainDB allows flexibility in this respect, remaining open to the many possible approaches that will undoubtedly be created by users.
BigchainDB therefore limits its role to providing a mechanism for reliable gathering and aggregation of signed data.

We imagine a rich community of authorities signing assets. 
For example, point-of-creation software or hardware vendors could certify that a particular digital creation was created by their software or hardware at a given point in time. 
Individuals could leave certified reviews for movies, restaurants, consumer goods, or even reputational reviews for other individuals.
Other examples could emerge from ideas in prediction markets \cite{hanson1993consensus}, the issuance of securities, and the rating of those securities.

In a model where anyone can issue an authoritative statement about someone or something, the reputation of the Authenticator will be critically important.
How do you know whose certificates you can trust?
We anticipate the development of social media reputation systems that go beyond Facebook Friends and Likes.
BigchainDB enables the widespread implementation of new reputation systems such as the Backfeed protocol for management of distributed collaboration \cite{backfeed}, or systems drawing inspiration from fictional reputation economies described by Cory Doctorow (Whuffie) \cite{doctorow2003down}, Daniel Suarez (Levels) \cite{suarez2010freedomtm}, and others.
