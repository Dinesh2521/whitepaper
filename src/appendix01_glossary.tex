\section{Glossary}\label{appendix:glossary}

\noindent\textbf{SQL DB} – a database that stores data in table format and supports the Structured Query Language (SQL); a relational database. Example: MySQL.

\medskip
\noindent\textbf{Ledger} – a database that stores data in a table format, where entries are economic transactions

\medskip
\noindent\textbf{NoSQL DB} – a database that stores data in a non-table format, such as key-value store or graph store. Example: RethinkDB.

\medskip
\noindent\textbf{NoQL DB} – a database without any query language to speak of. Obviously, this hinders data management. Example: Bitcoin blockchain.

\medskip
\noindent\textbf{Distributed DB} – a database that distributes data among more than one node in a network\footnote{Sometimes there is confusion, and “distributed” is used when the actual meaning is that of “decentralized”, most notably with the term “distributed ledger”.}. Example: RethinkDB.

\medskip
\noindent\textbf{Fully replicated DB} – a distributed DB where every node holds all the data.

\medskip
\noindent\textbf{Partially replicated DB} – a distributed DB where every node holds a fraction of the data.

\medskip
\noindent\textbf{Decentralized DB} – a DB where no single node owns or controls the network.

\medskip
\noindent\textbf{Immutable DB} – a DB where storage on the network is tamper-resistant.

\medskip
\noindent\textbf{Blockchain DB} – a distributed, decentralized, immutable DB, that also has ability for creation \& transfer of assets without reliance on a central entity.

\medskip
\noindent\textbf{Bitcoin blockchain} – a specific NoQL, fully-replicated, blockchain DB.

\medskip
\noindent\textbf{BigchainDB} – a specific NoSQL\footnote{There can be SQL support to via wrapping the NoSQL functionality or using the BigchainDB design on a distributed SQL DB}, partially-replicated, blockchain DB.